%!TEX root =  ../main.tex
\ctexset{
    chapter/format = {\centering\sffamily\bfseries\zihao{3}},
    section = {
        format     = \rmfamily\bfseries\zihao{-4}, %sffamily\setfontsize{14\bp@},
        aftername  = {、},
        beforeskip = 24bp,
        afterskip  = 6bp,
        number = \arabic{section}
    }
}
\providecommand\swjtuClass{通信20xx-xx班}
\providecommand\swjtuAuthor{张XX}
\providecommand\swjtuNo{2020xxxxxx}
\providecommand\swjtuTitle{示例标题}
\addtocontents{toc}{\protect\setcounter{tocdepth}{0}}
\rmfamily\zihao{-4}\linespread{1.5}\selectfont
\chapter{毕业设计(论文)任务书}
\vspace*{\baselineskip}
\noindent
\begin{tabular}%
    {@{}p{.35\linewidth}@{}>{\centering\arraybackslash}
    p{.3\linewidth}@{}>{\raggedleft\arraybackslash}p{.35\linewidth}@{}}
班\hspace{2\ccwd}级:\uline{\hspace*{2pt}\swjtuClass\hspace{\stretch{1}}}\hspace*{.5\ccwd} &
学生姓名:\uline{\hspace*{2pt}\swjtuAuthor\hspace{\stretch{1}}}\hspace*{.5\ccwd} &
学\hspace{2\ccwd}号:\uline{\hspace*{2pt}\swjtuNo\hspace*{\stretch{1}}}\\
\end{tabular}\\
发题日期:20xx年xx月xx日\hspace{\stretch{1}}
完成日期:20xx年xx月xx日\\
题\hspace{2\ccwd}目:\uline{\hspace{3pt}\swjtuTitle{}\hspace*{\stretch{1}}}\par

\section{本设计(论文)的目的、意义}
设计的目的是摆烂睡大觉。

\section{学生应完成的任务}
本部分为学生应完成的认为。

\section{本设计(论文)与本专业的毕业要求达成度如何?
        (如在知识结构、能力结构、素质结构等方面有哪些有效的训练。)}
按要求填写

\section{本设计(论文)各部分内容及时间分配
        (共{\hspace{1em}17\hspace{1em}}周)}
按要求填写

\vskip\baselineskip\noindent
备\hspace{2\ccwd}注:这里是备注

\vskip\baselineskip\noindent
指导教师:\uline{\hspace{7em}}
\hspace{5em}年\hspace{2em}月\hspace{2em}日\par

\mainchapterformat
\addtocontents{toc}{\protect\setcounter{tocdepth}{2}}